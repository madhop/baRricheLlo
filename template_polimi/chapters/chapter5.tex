\chapter{Theoretical Background}
\label{Theoretical Background}
\thispagestyle{empty}

TESTO CAPITOLO 5


- Reinforcement learning:
Machine learning (ML) is a process whereby a computer program learns from experience to improve its performance at a specified task [16]. ML algorithms are often classified under
one of three broad categories: supervised learning, unsupervised learning and reinforcement learning (RL). Supervised learning algorithms are based on inductive inference where the model is typically trained using labelled data to perform classification or regression, whereas unsupervised learning encompasses techniques such as density estimation or clustering applied to unlabelled data. By contrast, in the RL paradigm an autonomous agent learns to improve its performance at an assigned task by interacting with its environment.
RL agents are not told explicitly how to act by an expert; rather an agent's performance is evaluated by a reward function R. For each state experienced, the agent chooses an action and receives an occasional reward from its environment based on the usefulness of its decision. The goal for the agent is to maximize the cumulative rewards received over its lifetime. Gradually, the agent can increase its long-term reward by exploiting knowledge learned about the expected utility (i.e.discounted sum of expected future rewards) of different state-action pairs. One of the main challenges in reinforcement learning is managing the trade-off between exploration and exploitation. To maximize the rewards it receives, an agent must exploit its knowledge by selecting actions which are known to result in high rewards. On the other hand, to discover such beneficial actions, it has to take the risk of trying new actions which may lead to higher rewards than the current best-valued actions for each system state. In other words, the learning agent has to exploit what it already knows in order to obtain rewards, but it also has to explore the unknown in order to make better action selections in the future. Examples of strategies which have been proposed to manage this trade-off include 2 -greedy and softmax. When adopting the ubiquitous 2 -greedy strategy, an agent either selects an action at random with probability 0 < 2 < 1 , or greedily selects the highest valued action for the current state with the remaining probability 1 - 2 . Intuitively, the agent should explore more at the beginning of the training process when little is known about the problem environment. As training progresses, the agent may gradually conduct more exploitation than exploration.

- MDP

- FQI

	When the state and action spaces are finite and small enough, the Q-function can be represented in tabular form, and its approximation (in batch and in on-line mode) as well as the control policy derivation are straightforward. However, when dealing with continuous or very large discrete state and/or action spaces, the Q-function cannot be represented anymore by a table with one entry for each state-action pair. Moreover, in the context of reinforcement learning an approximation of the Q-function all over the state-action space must be determined from finite and generally very sparse sets of four-tuples.
To overcome this generalization problem, a particularly attractive framework is the one used by Ormoneit and Sen (2002) which applies the idea of fitted value iteration (Gordon, 1999) to kernel-based reinforcement learning, and reformulates the Q-function determination problem as a sequence of kernel-based regression problems. Actually, this framework makes it possible to take full advantage in the context of reinforcement learning of the generalization capabilities of any regression algorithm, and this contrary to stochastic approximation algorithms (Sutton, 1988; Tsitsiklis, 1994) which can only use parametric function approximators (for example, linear combinations of feature vectors or neural networks). In the rest of this paper we will call this framework the fitted Q iteration
algorithm so as to stress the fact that it allows to fit (using a set of four-tuples) any (parametric or non-parametric) approximation architecture to the Q-function.
The fitted Q iteration algorithm is a batch mode reinforcement learning algorithm which yields an approximation of the Q-function corresponding to an infinite horizon optimal control problem with discounted rewards, by iteratively extending the optimization horizon (Ernst et al., 2003).
At each step this algorithm may use the full set of four-tuples gathered from observation of the system together with the function computed at the previous step to determine a new training set which is used by a supervised learning (regression) method to compute the next function of the sequence. It produces a sequence of Q N -functions, approximations of the Q N -functions defined by Eqn (5). --> mettere algoritmo a pag 6 di tree based batch reinforcement learning

- extra trees
	Besides Tree Bagging, several other methods to build tree ensembles have been proposed that often improve the accuracy with respect to Tree Bagging (e.g. Random Forests, Breiman, 2001). In this paper, we evaluate our recently developed algorithm that we call "Extra-Trees", for extremely randomized trees (Geurts et al., 2004). Like Tree Bagging, this algorithm works by building several (M) trees. However, contrary to Tree Bagging which uses the standard CART algorithm to derive the trees from a bootstrap sample, in the case of Extra-Trees, each tree is built from the complete original training set. To determine a test at a node, this algorithm selects K cut-directions at random and for each cut-direction, a cut-point at random. It then computes a score for each of the K tests and
chooses among these K tests the one that maximizes the score. Again, the algorithm stops splitting a node when the number of elements in this node is less than a parameter n min . Three parameters are associated to this algorithm: the number M of trees to build, the number K of candidate tests at each node and the minimal leaf size n min . The detailed tree building procedure is given in Appendix A.
	
	- Double learning
	- Double FQI

- DDPG
	- DPG
	