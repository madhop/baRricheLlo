\chapter{State of the Art}
\label{State of the Art}
\thispagestyle{empty}




2.1 Reinforcement learning in autonomous driving

A combination of the advantages of both, the speed of the Riccati controller and the generality of MPC, can be achieved by finding a function that maps state values to
control variables, e. g., by training a deep neural network. Such a model could, for example, be learned supervised, as done for PILOTNET, or by reinforcement learning. The latter in particular led to excellent results in the training of such agents for controlling real-world systems such as robots or helicopters.
Recent work also shows promising applications of reinforcement learning for autonomous driving by making strategic decisions. 
Autonomous driving tasks where RL could be applied include: controller optimization, path planning and trajectory optimization, motion planning and dynamic path planning, development of high-level driving policies for complex navigation tasks, scenario-based policy learning for highways, intersections, merges and splits, reward learning with inverse reinforcement learning from expert data for intent prediction for traffic actors such as pedestrian, vehicles and finally learning of policies that ensures safety and perform risk estimation. Further, it turns out to be suitable in contexts of autonomous racing: the driverless racer could learn a policy that is able to outperform the performance of a human driver, or a policy taught by experts.
- Controlling an Autonomous Vehicle with Deep Reinforcement Learning
This paper https://arxiv.org/pdf/1909.12153.pdf shows an example of end-to-end reinforcement learning application to autonomous drive, using a proximal policy optimization algorithms by means of a neural network which maps the state to the controls. It learns two different policies: the driver and the stopper, and the reward is the squared proximity to a target position. The work aims at realizing the autonomous exploration of a parking lot based on deep reinforcement learning. It describes how a policy is trained to compute sophisticated control commands which depend on an estimate of the current vehicle state. This is done by designing an appropriate Markov decision process and a corresponding proximal policy optimization learning algorithm. For that purpose a simulated environment is used for data generation. Here, information about the vehicle's surrounding are measured by, e. g., laser scanners and are further extended by a rough knowledge about the geometry of the drivable area.
Here, the state is composed by: steering wheel angle and longitudinal acceleration; the state transition is made by single-track-models. The vehicle is assumed to have only one wheel at the front and back respectively, each centered between the real ones. Finally, two reward function are used in order to train a driver and a stopper function.


- CARMA: A Deep Reinforcement Learning Approach to Autonomous Driving
In this paper they experiment different setups of Deep Neural Networks, such as CNN and RNN as functions approximator of a reinfocement learning agent.
They formulate the state-space as follows: At each timestep, the agent receives an image I t of the environment, an estimate of the agent's speed, an estimate of the agent's distance to the center of the road, and an estimate of the angle between the agent's forward vector and the center of the road. At each timestep t, the agent can perform a discrete action: acceleration a = (Brake, DoNothing, Accelerate) and steer t = (TurnLef, DoNothing, TurnRight). Thus,there are a total of 9 possible actions the car can take at each timestep.
Then, they experimented different modeling approaches based on Q-Learning, involving CNN and RNN agents.


- Simulation-based reinforcement learning for autonomous driving:
This paper applies via reinforcement throughe Proximal Policy Optimization (PPO) with a contin-ous action space operating on multiple tensors of multiple shapes: a front camera, a high-level navigation command, car speed, and car acceleration. Thus, they adopted a custom policy that operates on multiple input tensors coming from one observation. They operated within the CARLA simulator. Their policy controls only the steering, while throttle is controlled via a PID controller with the speed set to a constant.
They modeled continuous actions with the Gaussian distribution.
The agent receives as its observation an RGB image from a single front camera from which extract semantic segmentation and car metrics such as speed and acceleration. The agent is also provided with high-level navigation command. 

	
3.

- Autonomous driving in racing:
Finding a racing line that allows to achieve a competitive lap-time is a key problem in real-world car racing as well as in the development of non-player characters for a commercial racing game.
The optimal racing line is defined as the line to follow to achieve the best lap-time possible on a given track with a given car. In general, finding the optimal racing line is
a common problem in real-world car racing as well as in the development of commercial racing games. The optimal racing line is the path that a driver should follow to complete a lap on a given track in the smallest amount of time possible. As the lap-time depends both on the distance raced and on the average racing speed, finding the optimal racing involves two different sub-problems: racing the shortest distance possible and racing as fast
as possible along the track.


Today there exist several racing-oriented driving simulators, along with developing environment. Some examples are AWS Deep Racer (Paper: DeepRacer: Educational Autonomous Racing Platform for Experimentation with Sim2Real Reinforcement Learning), TORCS, Racer (Development of a Car Racing Simulator Game Using Artificial Intelligence Techniques).

During the years, different attempts to achieve time-optimal racing, according with different technologies. 

- mettere una parte di come funziona l'intelligenza artificiale nei videogiochi


- Evolving the Optimal Racing Line in a High-End Racing Game
This paper shows how to encode a racing line by a set of connected Bézier curves, such that each gene defines a small portion of the racing line. Therefore, the evolution is responsible of the entire design of the racing line. In addition, the authors compare two different methods to evaluate the evolved racing line; the first one is based on testing the evolved racing lines in a racing simulator; the second one consists of estimating the performance of a racing line through a computational model.
	
- Ahura: A Heuristic-Based Racer for the Open Racing Car Simulator
This work proposes a controller called Ahura for TORCS. The controller uses five modules: steer controller, speed controller, opponent manager (it creates a map of opponents around  and finds the vacant slot to overtake), dynamic adjuster (friction, bumps), stuck manager (uses the idea proposed in to control the vehicle when it is out of the track or it has stuck somewhere.).
There are 23 parameters in Ahura that need to be determined:
1) eight parameters for the steer controller 
2) ten parameters for the speed controller 
3) five parameters for the opponent manager
They were determined by using the optimization algorithm CMA-ES: it has a good performance in continuous space, works with nonlinear systems, no constraint handling technique is required, and it is appropriate for nonseparable search spaces.


- Model predictive control Towards Time-Optimal Race Car Driving using Nonlinear MPC in
Real-Time
Model Predictive Control (MPC) is a multivariable control algorithm that uses:
an internal dynamic model of the process
a cost function J over the receding horizon
an optimization algorithm minimizing the cost function J using the control input u
Existing advanced control based attempts to minimum-time driving usually provide only offline open-loop solutions. . In this paper, instead, we directly use a one-level approach, solving the nonlinear MPC problem with an economic cost function in real-time. The tight real-time bounds imposed on computational times make it necessary to
reformulate the problem so as to allow for the use of efficient algorithms.
It provides real-world experimental results of the proposed nonlinear MPC scheme from a miniature race-car setup that is detailed in the paper. 


- Learning Drivers for TORCS through Imitation Using Supervised Methods	
In this work, Cardamone, Loiacono and Lanzi applied supervised learning to develop car controllers for The Open Car Race Simulator from the logs collected from other drivers. 
They considered two representations of the current state of the car: the set of rangefinder inputs usually employed in simulated car racing competitions, and a high-level, qualitative, representation involving basic lookahead information about the track in front of the car. Instead of predicting the typical low-level actions on the car actuators available in TORCS (namely, the steering wheel, the gas pedal, the brake pedal and the gear change), their approach predicts a target speed and the car position with respect to the track axis.
They considered two supervised learning methods, multi-layer neural networks and k-nearest neighbor classifiers, and applied them to compute a model mapping input sensors to actions which could imitate the behavior of an observed driver.

- Continuous Reinforcement Learning From Human Demonstrations With Integrated Experience Replay for Autonomous Driving


- Our work proposed different approaches to solve time-optimal racing problem in the framework of reinforcement learning. 











The objective of this chapter is to give an overview about what has been done in the field of autonomous driving putting its focus expecially on reinforcement-learning-based algorithms.
In particular, the chapter is divided in two sections. First, it describes the theoretical background in reinforcement learning.
In the second part we give insight about successful applications of reinforcement learning to autonomous driving.

\section{Theoretical Background}



\section{Related Works}



- parlare di alphago e atari

By learning a deep convolutional neural network
to approximate the Q-function, Mnih et al. successfully
construct a DRL framework called Deep Q-Network (DQN)
which plays Atari games in human level

Finn et al. proposed the Inverse Reinforcement
Learning (IRL), which automatically learns reward functions
from human demonstrations for robotic control [11].

Presented in [16], Deep Q-learning from Demonstrations
(DQfD) vastly accelerates DQN by pretraining an initial
behavior network, and also introducing a supervised loss and
a L2 regularization loss when training the target network.





- background:
pag 3-4 cinesi













- \cite{cinesi} Integrating
RL with human expertise is technically an interesting way
to accelerate the exploration and increase the stability. In
this work, we propose a continuous reinforcement learning
method which integrates Deep Deterministic Policy Gradient
(DDPG) with human demonstrations.
In this paper, we want the agent to learn both from the
demonstrator and the environment, which means that, if
the agent is currently at a certain state of demonstration
buffer, the action it takes should be consistent with the
corresponding demonstrator’s action. So, we evaluate the
similarity between these two actions by the following
formulation: combined loss


