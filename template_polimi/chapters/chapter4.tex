\chapter{State of the Art}
\label{State of the Art}
\thispagestyle{empty}



Gli algoritmi di guida autonoma esistono già, c'è questo quello quell'altro
	Prima parte: 
		guida autonoma
	Seconda parte:
		Reinforcement learning based self driving car
		Videogiochi guida autonoma
		ieee transaction on ai games
		check evolutionary intelligent gaming
		gameconf
		papers su pagina di lanzi
		
- Introduzione intelligenza artificiale:
	cos'è l'intelligenza artificiale
	perchè l'intelligenza artificiale
	robots
	autonomous vehicles
	
- Autonomous vehicles
	taxonomy
	albori
		rules
	oggi
		machine learning
			reinforcement learning
				fisico (tesla, google..)
				virtuale (videogiochi)
				 
	
	











Simulators: Voyage Deep Drive, CARLA, AWS Deep Racer, Deep Traffic





			


with the dramatic improvement in hardware and the availability of a huge amount of data machine learning and deep learning techniques could have been applied to this scenario. 
In particular, reinforcement learning has been demonstrated to be quite promising for this context.







\cite{playingforbenchmarks}






Introduzione al capitolo


The objective of this chapter is to illustrate more in detail the context in which our thesis is located.
In particular, the chapter is divided in two sections. First, it describes the work done by the scientific community in the field of autonomous driving, focusing expecially on reinforcement learning based algorithms.
In the second part we provide an overview of the world of autonomous driving videogames and simulators.





Introduzione alla guida autonoma

In the last decade autonomous driving has been deeply studied in several fields of research: automotive, military, racing cars, and even fly. The goal is to disrupt the need of a human driver or pilot, resulting in a reduction in costs and an increase of safety, for instance by reducing the number of accidents per year due to human error. Moreover, they could learn from their errors, and transmit to each other autonomous vehicle instantaneously, so that each of them is constantly up to date.
Researches for autonomous driving traces back to the birth of the first computers, in the '40s. At that moment, the scientific community were trying and gain insight about the human mind, in order to build artificial agents that could replicate -or even outperform- human in some specific tasks.
That's how computer science branched into paths such as artificial intelligence and robotics.
One of the main goal of the research of artificial intelligence has always been autonomous driving, that is the capability for an agent of perceiving the world and moving accordingly in order to reach a specific target.
Autonomous cars consists in some piece of hardware capable of moving, such as a traditional car, or a drone, equipped with some sensors give insight on the environment, one or more computer units whith a decision making algorithm, which compute an action to perform by means of some actuators, such as a throttle, a break, a steering, or a transmission.
One of the first experiments in embodying some intelligence in an artificial agent were William Grey Walter's Turtle Robots in the '50s. They were capable of moving in the sorroundings by sensing the environment in a simplified manner. They consisted in front wheel drive tricycle-like robots covered by a clear plastic shell, and were provided with a photocell and a bump detector as sensors, which resulted in the action of a motor. Despite their simple behaviors, the technique Walter used are reflected in today’s reactive and biologically-inspired robots such as those based on the B.E.A.M philosophy.
Later on, in 1989, a new tentative of building an autonomous car was ALVINN, which stands for Autonomous Land Vehicle In a Neural Network, developed for military research.
At that time the technology was not sophisticated enough to provide the computation required to drive in real time, but in a sense the premise of the algorithm used nowadays was already there. In fact, neural networks are today the essential tools for building an autonomous car.
Later on with the advent of more and more sophisticated electronic components, such as sensors (Lidars and Radars, which are capable of scanning the environment at 360 degree via electromagnetic waves), and with the continuous improvement of electronic components, such as GPUs, autonomous driving started being a hot topic in industry beside scientific research.
In the last decade, most of the traditional automotive companies, expecially Mercedes-Benz, Volkswagen, Volvo, started investing in this reasearch providing their cars with a multitude of sensors and algorithms to make their cars autonomous. Ford is another manufacturer with deployments already in play, with self-driving vehicles being tested in Pittsburgh, Palo Alto, Miami, Washington D.C. and Detroit, with Austin, Texas joining them soon. Together with its partner Argo AI, Ford has plans to trial its fleet of self-driving cars in Austin with a view towards launching a wider-reaching autonomous taxi and delivery service in 2021.
Tesla, one of the companies founded by Elon Musk, is also making big steps forward in taking autonomy into mainstream use, both in terms of real world use cases and potential monetization of self-driving technologies. Tesla has supplied customers with more than 780,000 vehicles since launching, the majority of which arrive with pre-installed, self-driving capabilities available to users who purchase the requisite software. Tesla autonomous vehicles have logged huge levels of miles driven since their introduction, growing from 0.1 billion miles in May 2016 to an estimated 1.88 billion miles as of October 2019.
Waymo, the newborn firm from Google's Alfabet, has been carrying out successful trials of autonomous taxis in California, transporting over 6,200 people in the first month and many thousands since. They're proving a practical business case for autonomous vehicles.
Also in the U.S., Walmart is using autonomous cargo vans to deliver groceries in Arizona, while Pizza Hut is working with Toyota on a driverless electric delivery vehicle that even has a mobile kitchen in it to cook pizzas en route to your house.
Parallely, in the military field, DARPA (Defense Advanced Research Projects Agency) has been proposing every year since 2007 a challenge called DARPA Grand Challenge, in which scientist teams dare each other to reach some targets as fast as possible.
The reasons for this race to the autonomous driving are millions of possible accidents avoided per year, and a reduction of pollution and costs by sharing cars which are capable of transporting people without human intervention needed.
This growth in the reaserach on autonomous cars led to a formalization of the levels of automation: Society of Automotive Engineers (SAE) introduced five levels of automation: with Level 1, driver assistance, relating to computer assistance of simple driving functions like the cruise control or automated braking systems. Cruise control consists in the capability of mantaining a certain target speed, whatever the slope of the road, weather condition, asphalt roughness. It can be accomplished via some speed sensors and a PID controller, there is no need of complex artificial intelligence. Automated braking systems involves stopping the car or reducing the speed whenever an object come across, be it a vehicle or a pedestrian. It requires proximity and distance sensors (ultrasonic or laser sensors) and may follow some manual rules based on thresholds. Lane Crossing Alert makes the car capable of notify whenever it crosses another lane, and can be achieved with camera sensors.
Level 2 refers to partial automation, where the vehicle assists drivers with steering or acceleration, allowing drivers to disengage from some tasks.  For example instance , Adaptive Cruise Control and Lane Keeping.
Level 3 concerns conditional automation, where the vehicle takes over some of the monitoring of the environment from the driver, using sensor technology like LiDAR. That's what the company Tesla is currently developing: their cars are able of moving in the surroundings but the human intervention is still required in dangerous situations.
Level 4 is high automation, where much greater control has been handed to the vehicle, which is in charge of steering, braking, accelerating, monitoring the vehicle and roads, and also responding to events like deciding when to change lanes, turn or use signals.
Level 5 is full automation. No company is currently able to reach this level of automation.
Currently, most of the cars are embodied with level 1 or 2. However, level 3 and 4 are still object of research, especially by tech companies such as Tesla and Waymo, whose promise is to reach these levels of automation in ten years, and level 5 in twenty, enabling their cars with the power of fully replacing human drivers.
The reasons why today full autonomous driving has not being implemented yet are the extremely huge amount of data required for perceiving the complex world of the urban scenario and for computing the consequent actions. In fact, in order to be able to perform this vast computations, several computers and GPUs are needed aboard on the car, resulting in cost, weight, and power consumption. This is the strategy adopted by Tesla so far, whose cars to day span from a price of 85000 dollars to 120000 dollars, and weigh about 2000kg. 
Another approach, adopted by companies such as Google, is to perform computation on remote servers in their datacenters. However, current mobile connections such as 4G, makes impractical the transmission of data provided each second by the vast amount of sensors.
In the future, the advent of 5G could be a game changer in this sense, which should provide a larger in bandwidth and more stable internet connection.
That's why the scientific community is starting downstepping the complexity of the task, trying to build autonomous agents in a closed and controlled environment. This lead to a simplification of the problem, avoiding the need of a real time mapping of the environment, and excluding unattended events such as pedestrian coming across.



In the last decade, automotive and tech companies are struggling to develop full autonomous driving cars. However, wether it's a matter of hardware or software, today such goal is still out of reach. Rather, some firms are focusing their attention on making a car which is autonomous in a specifing context. For example, good results have been achieved in keeping a lane on a highway, or stopping with a pedestrian coming across. 
In this paper we will tackle the problem of following a trajectory driven beforehand by a human driver on a race car and, if possible, to improve it.




	- 5 su ambienti controllati: piste formula 1 per test, videogiochi
	
	
Come si fa la guida autonoma

Automation to this level is a rather complex task: it requires a part of perception of the environment, a part of planning through an algorithm, and a part of control, to perform the actions by means of physical actuators.
In our thesis we focused our attention on the planning part. In fact, by the help of a driving simulator, we could gather data with help of virtual sensors, and control the car by virtual actuators.



	- paradigma sense, plan, act
	- sistemi operativi: ros, apex.os
	- architettura: 
		- sense: 
			- hardware: lidar, gps, odometry, radar
			- software: vision, segmentation, deep learning, thresholding ecc
		- plan:
			- hardware: computer, gpu, 5G
			- software: 
				- edge computing o local computing
				- reinforcement learning: policy, reward ecc
					- papers sul reinforcement learning applicati alle macchine
				- pid non suitable perchè il mondo non è noto a priori
		- act:
			- steer, throttle, brake...

Dove si fa la guida autonoma
	- urban
	- consegne
	- formula 1: roborace, porsche, our project



.Reinforcement learning autonomous driving:
	- cenni a rl e perchè è suitable a questa applicazione:
			- mondo non noto a priori
			- impara sempre
			- impara ad adattarsi
	- cosa è stato fatto
		- 
- This paper https://arxiv.org/pdf/1909.12153.pdf shows an example of end-to-end reinforcement learning application to autonomous drive, using a proximal policy optimization algorithms by means of a neural network which maps the state to the controls. - It learns two different policies: the driver and the stopper, and the reward is the squared proximity to a target position.
In this paper $https://web.stanford.edu/~anayebi/projects/CS_239_Final_Project_Writeup.pdf$ they experiment different setups of Deep Neural Networks, such as CNN and RNN as functions approximator of a reinfocement learning agent.
- Lane keeping rl https://www.sae.org/publications/technical-papers/content/2020-01-0728/
- 


.Videogiochi e simulation autonomous driving:
	- ambienti di sviluppo:  Voyage Deep Drive, CARLA, AWS Deep Racer, Deep Traffic, TORCS	
	- rl applicati a simulatori e videogiochi:
		- https://deepsense.ai/wp-content/uploads/2019/06/Simulation-based-reinforcement-learning-for-autonomous-driving.pdf
	
	
	
	
	
	
