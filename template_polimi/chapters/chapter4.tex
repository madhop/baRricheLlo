\chapter{State of the Art}
\label{State of the Art}
\thispagestyle{empty}



Gli algoritmi di guida autonoma esistono già, c'è questo quello quell'altro
	Prima parte: 
		guida autonoma
	Seconda parte:
		Reinforcement learning based self driving car
		Videogiochi guida autonoma
		ieee transaction on ai games
		check evolutionary intelligent gaming
		gameconf
		papers su pagina di lanzi
		
- Introduzione intelligenza artificiale:
	cos'è l'intelligenza artificiale
	perchè l'intelligenza artificiale
	robots
	autonomous vehicles
	
- Autonomous vehicles
	taxonomy
	albori
		rules
	oggi
		machine learning
			reinforcement learning
				fisico (tesla, google..)
				virtuale (videogiochi)
				 
	
			
In the last decade autonomous driving -and robotics in general- has been deeply studied in several fields of research. 
Ever since the born of the first computers, the scientific community tried to gain insight about the human mind, in order to replicate it -or even outperform it- onto artificial agents that could 


Universities and big tech companies are putting their efforts to compete in the race for the best driverless car. 



In order to classify Autonomous Vehicles, a specific SAE taxonomy has been introduced (International, 2016). It identifies six levels of progressive automation: from 0 to 2 they
involve up to partial automation of the vehicle and decisions are under the control of the
driver. Levels 3 to 5 lead to an increasing automation level, culminating into the full
automation of level 5 where the driver is overridden by the system.



with the dramatic improvement in hardware and the availability of a huge amount of data machine learning and deep learning techniques could have been applied to this scenario. 
In particular, reinforcement learning has been demonstrated to be quite promising for this context.




In 2020, you’ll be a “permanent backseat driver,” the Guardian predicted in 2015. “10 million self-driving cars will be on the road by 2020,” blared a Business Insider headline from 2016. Those declarations were accompanied by announcements from General Motors, Google’s Waymo, Toyota, and Honda that they’d be making self-driving cars by 2020. Elon Musk forecast that Tesla would do it by 2018 — and then, when that failed, by 2020.

Since 2015, automotive companies are struggling to develop full autonomous driving cars. However, wether it's a matter of hardware or software, today such goal is still out of reach. Rather, some firms are focusing their attention on making a car which is autonomous in a specifing context. For example, good results have been achieved in keeping a lane on a highway, or stopping with a pedestrian coming across. 
In this paper we will tackle the problem of following a trajectory driven beforehand by a human driver on a race car and, if possible, to improve it.



