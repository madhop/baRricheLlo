\chapter{State of the Art}
\label{State of the Art}
\thispagestyle{empty}



Gli algoritmi di guida autonoma esistono già, c'è questo quello quell'altro
	Prima parte: 
		guida autonoma
	Seconda parte:
		Reinforcement learning based self driving car
		Videogiochi guida autonoma
		ieee transaction on ai games
		check evolutionary intelligent gaming
		gameconf
		papers su pagina di lanzi
		
- Introduzione intelligenza artificiale:
	cos'è l'intelligenza artificiale
	perchè l'intelligenza artificiale
	robots
	autonomous vehicles
	
- Autonomous vehicles
	taxonomy
	albori
		rules
	oggi
		machine learning
			reinforcement learning
				fisico (tesla, google..)
				virtuale (videogiochi)
				 
	
	

Since computers were born, the scientific community started wondering wether it is possible to emulate - and if possible, outperform - human behaviors and, in general, human mind. That's how artificial intelligence and robotics were born. In this field autonomous cars have lately become one of the main object of research. They promise to have an impact on safety, leading to a decrease of millions of road accidents per year.

Driverless cars consists in a traditional car equipped with a multitude of sensors which provide a representation of the environment, one or more computers whith a decision making algorithm installed, which compute the best action to take, which is performed to some actuators.

Since 2004, each year DARPA publishes the DARPA Grand Challenge, a challenge in which scientific teams compete with their autonomous vehicles at reaching specific goals.
All the major tech and automotive companies are investing in driverless cars, such as Alfabet's Waymo, Tesla, Chrysler etc.

Automation to this level is a rather complex task: it requires a part of perception of the environment, a part of planning through an algorithm, and a part of control, to perform the actions by means of physical actuators.
In our thesis we focused our attention on the planning part. In fact, by the help of a driving simulator, we could gather data with help of virtual sensors, and control the car by virtual actuators.


The objective of this chapter is to illustrate more in detail the context in which our thesis is located.
In particular, the chapter is divided in two sections. First, it describes the work done by the scientific community in the field of autonomous driving, focusing expecially on reinforcement learning based algorithms.
In the second part we provide an overview of the world of autonomous driving videogames and simulators.



1. 

This paper https://arxiv.org/pdf/1909.12153.pdf shows an example of end-to-end reinforcement learning application to autonomous drive, using a proximal policy optimization algorithms by means of a neural network which maps the state to the controls. It learns two different policies: the driver and the stopper, and the reward is the squared proximity to a target position.
In this paper $https://web.stanford.edu/~anayebi/projects/CS_239_Final_Project_Writeup.pdf$ they experiment different setups of Deep Neural Networks, such as CNN and RNN as functions approximator of a reinfocement learning agent.


2. 



Simulators: Voyage Deep Drive, CARLA, AWS Deep Racer, Deep Traffic





			
In the last decade autonomous driving -and robotics in general- has been deeply studied in several fields of research. 
Ever since the born of the first computers, the scientific community tried to gain insight about the human mind, in order to replicate it -or even outperform it- onto artificial agents that could 


Universities and big tech companies are putting their efforts to compete in the race for the best driverless car. 



In order to classify Autonomous Vehicles, a specific SAE taxonomy has been introduced (International, 2016). It identifies six levels of progressive automation: from 0 to 2 they
involve up to partial automation of the vehicle and decisions are under the control of the
driver. Levels 3 to 5 lead to an increasing automation level, culminating into the full
automation of level 5 where the driver is overridden by the system.



with the dramatic improvement in hardware and the availability of a huge amount of data machine learning and deep learning techniques could have been applied to this scenario. 
In particular, reinforcement learning has been demonstrated to be quite promising for this context.





Since 2015, automotive companies are struggling to develop full autonomous driving cars. However, wether it's a matter of hardware or software, today such goal is still out of reach. Rather, some firms are focusing their attention on making a car which is autonomous in a specifing context. For example, good results have been achieved in keeping a lane on a highway, or stopping with a pedestrian coming across. 
In this paper we will tackle the problem of following a trajectory driven beforehand by a human driver on a race car and, if possible, to improve it.




\cite{playingforbenchmarks}

























.Guida autonoma:
Cos'è la guida autonoma:
- breve storia guida autonoma:
	- robotica tartarughe
	- 1980 first attempts
- guida autonoma oggi:
	- 5 livelli di autonomia
	Society of Automotive Engineers (SAE) to explain the various levels of autonomy, with Level 1, driver assistance, relating to computer assistance of simple driving functions like the cruise control or automated braking systems. Level 2 refers to partial automation, where the vehicle assists drivers with steering or acceleration, allowing drivers to disengage from some tasks. Level 3 concerns conditional automation, where the vehicle takes over some of the monitoring of the environment from the driver, using sensor technology like LiDAR. Level 4 is high automation, where much greater control has been handed to the vehicle, which is in charge of steering, braking, accelerating, monitoring the vehicle and roads, and also responding to events like deciding when to change lanes, turn or use signals.
	- dove siamo arrivati:
	    - 1 e 2 ok, 3 e 4 in test
		- waymo, mercedes, volkswagen, tesla, approcci di ciascuno
		su strada
		- darpa: nella difesa
	- cosa abbiamo raggiunto: 
	Autonomous vehicles are gradually finding their way onto our roads. Earlier this year, Waymo, the self-driving unit of Google's sister-company Alphabet, carried out successful trials of autonomous taxis in California, transporting over 6,200 people in the first month and many thousands since. They're proving a practical business case for autonomous vehicles.
Also in the U.S., Walmart is using autonomous cargo vans to deliver groceries in Arizona, while Pizza Hut is working with Toyota on a driverless electric delivery vehicle that even has a mobile kitchen in it to cook pizzas en route to your house.
Tesla is also making big steps forward in taking autonomy into mainstream use, both in terms of real world use cases and potential monetization of self-driving technologies. Tesla has supplied customers with more than 780,000 vehicles since launching, the majority of which arrive with pre-installed, self-driving capabilities available to users who purchase the requisite software. Tesla autonomous vehicles have logged huge levels of miles driven since their introduction, growing from 0.1 billion miles in May 2016 to an estimated 1.88 billion miles as of October 2019.

Ford is another manufacturer with deployments already in play, with self-driving vehicles being tested in Pittsburgh, Palo Alto, Miami, Washington D.C. and Detroit, with Austin, Texas joining them soon. Together with its partner Argo AI, Ford has plans to trial its fleet of self-driving cars in Austin with a view towards launching a wider-reaching autonomous taxi and delivery service in 2021.

	- sfide:
		- guida in città
		- gestione mole di dati perception
		- expensive lidars
		
- cosa si puo fare oggi:
	- 1 e 2 su strada: 
	- semplici task: consegne
	- 5 su ambienti controllati: piste formula 1 per test, videogiochi
	
	
Come si fa la guida autonoma
	- paradigma sense, plan, act
	- sistemi operativi: ros, apex.os
	- architettura: 
		- sense: 
			- hardware: lidar, gps, odometry, radar
			- software: vision, segmentation, deep learning, thresholding ecc
		- plan:
			- hardware: computer, gpu, 5G
			- software: 
				- edge computing o local computing
				- reinforcement learning: policy, reward ecc
					- papers sul reinforcement learning applicati alle macchine
				- pid non suitable perchè il mondo non è noto a priori
		- act:
			- steer, throttle, brake...

Dove si fa la guida autonoma
	- urban
	- consegne
	- formula 1: roborace, porsche, our project



.Reinforcement learning autonomous driving:
	- cenni a rl e perchè è suitable a questa applicazione:
			- mondo non noto a priori
			- impara sempre
			- impara ad adattarsi
	- cosa è stato fatto
		- 
This paper https://arxiv.org/pdf/1909.12153.pdf shows an example of end-to-end reinforcement learning application to autonomous drive, using a proximal policy optimization algorithms by means of a neural network which maps the state to the controls. It learns two different policies: the driver and the stopper, and the reward is the squared proximity to a target position.
In this paper $https://web.stanford.edu/~anayebi/projects/CS_239_Final_Project_Writeup.pdf$ they experiment different setups of Deep Neural Networks, such as CNN and RNN as functions approximator of a reinfocement learning agent.


.Videogiochi autonomous driving:

