\chapter{Methods}
\label{Methods}
\thispagestyle{empty}

%TORCS is a well-known car racing simulator. There are ten special bots in TORCS that are controllable through network ports by a client (controller). There are some (virtual) sensors connected to these bots that observe the environment and send the information to the controller. There are 19 proximity sensors (the value of the ith sensor is shown by dist i ) in front of the bot with predefined angles that provide the distance between the vehicle and the edges of the track toward their angle. The range of these sensors (shown by max D in this paper) is 200 m in the current version of TORCS. The angle of the proximity sensor i is shown by angle i , and it is a value in [-90 ◦ , 90 ◦ ]. These angles can be set at the beginning of the simulation. The index of the proximity sensor in front of the vehicle is 0 (called the zero sensor) and the index of the other sensors is set from -9 to 9. In the rest of this paper, we assume angle x = angle -x . There are 36 opponent sensors (the value of the ith sensor is shown by opp i ) that are only sensitive to opponents. These sensors are all evenly distributed around the bot with every 10 ◦ (the index of the sensor in front is 18) and their range is 200 m. There  s a track position sensor (the value is shown by trackPos) that provides a real value in the interval [-∞, ∞] (∞ refers to the maximum value of the type double in computer), where -1 represents the right side and 1 represents the left side of the track and other values translate to out of the track. There are four wheel spin sensors (one for each wheel) that calculate the speed of the wheels spin. The value for the wheel i is shown by v i 3 and it is in m/s. There is an rpm sensor that provides the rotation per minutes of the engine and provides a real number in [0, 10 000]. The current gear is an integer in {-1, 0, . . . , 6} (-1 is the rear gear and 0 is the neutral) that is also provided. There are three sensors for the current speed of the vehicle along its front (the value is shown by xSpeed), sides (the value is shown by ySpeed), and above (the value is shown by zSpeed). Finally, there is a sensor that calculates the current damage of the vehicle that is a real number in [0, 10 000].%
%A controller should provide appropriate values for the following actuators: acceleration pedal (shown by accelPedal in this paper), braking pedal (shown by brakePedal in this paper), and clutch pedal (shown by clutchValue in this paper) that are real values in [0, 1], gear (shown by gearValue in this paper) that is an integer in {-1, 0, . . . , 6}, and steer (shown by steerValue in this paper) that is a real value in [-1, 1], corresponding to full right and full left. The decision by the controller should be made within the simulation time interval (set to 22 ms in the version 1.3.4 of TORCS). Hence, if the designed controller is slow, then it might act by some delays which may cause inappropriate movements.%